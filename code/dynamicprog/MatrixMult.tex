\documentclass[12pt]{article}
\usepackage{amsmath}
\usepackage{parskip}

\title{Matrix multiplication: dynamic programming}
\author{Amanda House, Matthew Gharrity}
\date{July 4, 2017}

\begin{document}

\maketitle

\textbf{Problem statement}

Given a matrix multiplication $A_1 \times A_2 \times \cdots \times A_s$,
where $A_i$ has dimensions $m_i \times n_i$,
and $n_i = m_{i +1}$
for $i \in [1, s)$,
find the minimal number of integer multiplications needed to evaluate the matrix product. You may place parentheses anywhere.

\textbf{Approach}

Note the cost of multiplying $A_{n \times m} \times A_{a \times b}$
where $m = a$
is on the order of $n \times m \times b$.

First, consider the base case. If there is one matrix, evaluating the matrix product of the matrix and itself requires zero multiplications.

Then, define and solve the subproblem. Consider a set of $x$ matrices. Assume the optimal answer for every contiguous sequence of length 0 to $x-1$ matrices is already known. With this information, find the optimal for all $x$ matrices. The approach can be thought of as finding the boundary $i$ which minimizes the costs of multiplying matrices within the group 0 to $i$, within the group $i+1$ to $x$, and multiplying at the boundary between the two.

Thus, the dynamic programming solution is:
\[
  dp[l][r]=
    \begin{cases}
      0 & \text{if}\ l=r \\
      \min\limits_{l \leq i < r} (n_l \cdot m_i \cdot n_r + dp[l][i] + dp[i+1][r]) & \text{otherwise}
    \end{cases}
\]

Lastly, consider the order in which to fill the array $dp$. For example, when solving for $dp[0][3]$, the value for $dp[2][3]$ must be known. This requires the algorithm to proceed diagonally in the array. In index arithmetic, this means the algorithm must solve for all $l$ and $r$ that have a difference of 1, then all $l$ and $r$ with a difference of 2, and so on.

\textbf{Solution}

The code is in MatrixMult.java.

\end{document}
